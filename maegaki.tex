\section*{謝辞}
\begin{center}
この本を読んでくださる方に \\
気力をくれる友人に \\
大切な人に \\
感謝と本書をささげます
\end{center}

\section*{前書き}

本書は、メールサーバのセキュリティ設定に必要な概念について説明をします。

Postfixをはじめとするメールサーバをあつかうとき、前提条件となる知識は色々あります。
じれが大事なことなので、前著に引き続き、本書でも書きます。
メールサーバの設定では、なぜSubmissionポートが必要なのか、なぜTLSを使うのか、そういうった前提知識が必要となります。

また、メール関連のセキュリティは、自分を守るものもありますが、第三者に迷惑をかけない、第三者に判定の材料を提供する、という性質があります。

本書は、メール特有の、そういう概念にスポットを当てたものです。


\section*{本書の内容}
本書は、メールのセキュリティ設定に関連する概念ついて解説するものです。

\paragraph{第一章}
TLSによる通信の保護を、メールではどのように利用しているかについて解説します。
TCP上のアプリケーションにおけるTLSの利用方法について説明し、それがSMTPでどのように使用さr手いるかについて説明をします。

\paragraph{第二章}
メールサーバは、不特定多数のホストから接続を受けなければなりません。
ですが、spamの送信元など、招かれざる客を接続しないための方法も必要になります。
この章では、その方法について説明をします。


\paragraph{第三章}
メールの送信者偽装を検出する、ドメイン認証について説明します。まず、ドメイン認証という概念について説明祖行います。次に、ドメイン認証の規格について解説をします。


\section*{免責事項}
本書に書いてあることは、筆者知識のレベルでまとめたものです。ですが、内容が正しいとは言い切れません。これまでの本でも相当やらかしています。また、学校のレポート、業務などのコードを書く際に、本書の内容を信じて書いて損害が生じても、筆者にその責任はありません。

くれぐれも、自己責任と十分な検証の上、ご利用ください。

\section*{表紙イラスト}
ゆうちゃん (コース英知)