\section*{謝辞}
\begin{center}
この本を読んでくださる方に \\
気力をくれる友人に \\
大切な人に \\
感謝と本書をささげます
\end{center}

\section*{前書き}

本書は、メールサーバを扱うときに、前提知識となりがちな部分や、メールサーバに関連する者の、あまり語られることがなかった項目を、補講としてまとめたものです。

Postfixをはじめとするメールサーバをあつかうとき、前提条件となる知識は色々あります。
メールサーバは、OSの上で、各種のミドルウェアと共に動くことでサービスを提供します。
そのため、メールシステムは、サーバ、ネットワーク、ハードウェア、といった分野との関わりを持ちます。
ですが、その関わりの部分について説明される機会はあまりありません。・

本書は、そういう部分にスポットを当てて解説することを目的としています。
決して、締切り前のいい感じのところで中の人が風邪を引いたとか、そういうことはありません。ありませんよ。


\section*{本書の内容}
本書は、メールシステムを構築するときに必要となる、OSや他のミドルウェアとの関わりについて解説するものです。

\paragraph{第一章}
メールを転送するとはどういうことか、という観点から、メールリレー、メールルーティングについて解説しています。
メール転送はTCP/IPでいういところの亜婦ケーション層の通信であることと、DNSのエントリをどのようにようしているか、などの説明を行います。

\paragraph{第二章}
Postfixを動かすホストでのストレージの問題について解説します。
Postfixは、どのような用途でストレージを使用するかと、パフォーマンスとディスクI/Oの関係について記述しています。

\paragraph{第三章}
メールサーバの冗長化について説明します。受信メールサーバの。DNS連携によるレガシーな冗長化について解説します。

その次に、最終的にユーザが受信するメールを保存するサーバの冗長化について解説します。

\paragraph{第四章}
メールサーバを動かす環境として、オンプレミス、ハイパーバイザによる仮想マシン、コンテナを比較します。それぞれのメリット、デメリットに着いての検討を行います。

\section*{免責事項}
本書に書いてあることは、筆者知識のレベルでまとめたものです。ですが、内容が正しいとは言い切れません。これまでの本でも相当やらかしています。また、学校のレポート、業務などのコードを書く際に、本書の内容を信じて書いて損害が生じても、筆者にその責任はありません。

くれぐれも、自己責任と十分な検証の上、ご利用ください。

\section*{表紙イラスト}
ゆうちゃん (コース英知)