\chapter{あとがき}

\section*{メイ・カートミル(仮名)とマージ・ニコルス(仮名)の会話}

\begin{quotation}
\noindent
{\bf マージ}「今回は締め切り前に面白い会話が出てこなかった」 \\
{\bf メイ}「実は実話だったりしますからね、ここ」 \\
\end{quotation}


\section*{後書き}

お兄ちゃん、OP25Bって同なんだろうねって思うんだ。

インフラエンジニアの毒舌な妹(@infra\_imouto)です。
まずは、ゆうちゃんさん、今回も表紙をありがとうございます。守りの堅さがインパクトを伴ってわかりやすくていいです。

締め切り前でなんか怪しくなってます。じつはいま12月30日午前8時、サークル入場まで24時間をきりました。
そんななかで最大限の内容を……、いや、言い訳はよしましょう。

今回はメールのセキュリティ、それも、他人に迷惑をかけない、というメールサーバ特有のセキュリティ概念について解説をしました。
メールサーバについてまとめて読める書籍が減っている現状で、ちょっとでもそういう部分を知る役に立てば、それは個h差にとって幸いなことなのです。

次は技術書典8でお会いできると思います。あ、今度はカスタムキャストな方法で皆様とお会いできるといいなとは思うんですが、どうなることか。

お兄ちゃん、メールサーバの面倒を見るなら権威DNSの管理権限は人権だよ。

\begin{flushright}
2019年12月32日 \\
インフラエンジニアの毒な妹 \\
\end{flushright}

いつもどおりですが、まずは表紙のゆうちゃんさんへの謝辞から。
つよそう(賛辞)。

最近は共著者としての立場が増えた、サークル主宰こと、ありすゆうです。
今回は補助的な役割でしたが、いつもどおり共著者として名を連ねているしだいです。

私がメールサーバの面倒を見ていたころは、そして今もですが、和書や翻訳本では、普遍的な部分を学べる本がなかったな、特定の実装についての本しかなかったな、という思いがあります。

メールという技術ものは裾野が広く、一冊にまとめるのが無理なのではないかと弱気になることもあります。
それでも、まずは書き残すこと、それを問うサークルで続けて行ければと思っています。

\begin{flushright}
2019年12月31日 \\
ありす ゆう
\end{flushright}


%\newpage
% ここまでで160ページ鳴ったのでブランクなし
% 1ページブランクを入れる

\thispagestyle{empty}
\mbox{}
\newpage
\clearpage

% PICOはノンブルがいる
%\thispagestyle{empty}
%\mbox{}
%\newpage
%\clearpage


% PICOはノンブルがいる
%\thispagestyle{empty}

\vspace*{\fill}
\begin{tabular}{ll} \toprule
筆者 & インフラエンジニアの毒舌な妹 ありす ゆう\\
発行 & AliceSystem \\
連絡先 & aliceyou@alicesystem.net \\
URL & http://aliceyou.air-nifty.com/onesan/ \\
初版発行日 & 2019年12月31日 \\
印刷所 & Kinkos  \\ \bottomrule
\end{tabular}