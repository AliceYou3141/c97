\chapter{あとがき}

\section*{メイ・カートミル(仮名)とマージ・ニコルス(仮名)の沼津での会話}

\begin{quotation}
\noindent
{\bf メイ}「マージさん、干物を買ったらラブライブサンシャインの袋に入れて貰えましたよ」 \\
{\bf マージ}「……メイの私服、コスプレと思われておまけされた」 \\
{\bf メイ}「そんなにコスプレに見えますか?」 \\
{\bf マージ}「おばちゃんからみたら黒ニーソはヨハネのコスプレになってる可能性」 \\
\end{quotation}


\section*{後書き}

お兄ちゃん、メールサーバって関連することがらが思ったよりも多いよね。

インフラエンジニアの毒舌な妹(@infra\_imouto)です。
まずは、ゆうちゃんさん、今回も表紙をありがとうございます。今度のお姉さんは、Postfix本の表紙の姉妹の長女、という設定です。

今回は、メールというサービスを提供するシステムを構築するのに必要な、細かい点を解説した本となりました。そのため、下巻ではなく補講、という形にしています。
実際のところ、メールサーバは関連する技術分野が広く、ネットワークやサーバそのものの知識を必要とします。

DNSとの連携はその例として語られることが多いです。
そして、それ以外、どんなインスタンスで動かすか、どんなストレージ構成とするか、そう言った部分は、表に出ないノウハウとして秘匿さてきた面があります。

そういう部分を書いていくと、必要な知識ながらメールシステム構築というストーリーの中では収まりが悪い、という内容となりました。
それを独立した一冊に纏めた……と言いたいところですが、中の人が印刷所の締切り前で風邪を引いたためという経緯もあったり無かったり。

お兄ちゃん、ディスクリードの排他制御がサービスに与える影響を過小評価しすぎ。

\begin{flushright}
2019年8月12日 \\
インフラエンジニアの毒な妹 \\
\end{flushright}

いつもどおりですが、まずは表紙のゆうちゃんさんへの謝辞から。
年上ヒロイン属性の私としては、イイ、と一言。

最近は共著者としての立場が増えた、ありすゆうです。
今回はネタ出しを担当しました。
私も、キャンパスネットワークのレベルですが、長らく	Postfixのメールサーバの面倒を見てきました。

特に、ストレージの章については、あまり明文化する機会が無かった内容なので、短くも参考にしていただけるのではないかと思っています。あと、あえてか書かなかったけど、SATAのHDD積んでる安いサーバをメールサーバとして使うのはお勧めしません。

次はPostfixプラスプラスのリニューアルで、また補講になるのかな、と思いつつ。

\begin{flushright}
2019年8月12日 \\
ありす ゆう
\end{flushright}


%\newpage
% ここまでで160ページ鳴ったのでブランクなし
% 1ページブランクを入れる

%\thispagestyle{empty}
%\mbox{}
%\newpage
%\clearpage

% PICOはノンブルがいる
%\thispagestyle{empty}
%\mbox{}
%\newpage
%\clearpage


% PICOはノンブルがいる
%\thispagestyle{empty}

\vspace*{\fill}
\begin{tabular}{ll} \toprule
筆者 & インフラエンジニアの毒舌な妹 ありす ゆう\\
発行 & AliceSystem \\
連絡先 & aliceyou@alicesystem.net \\
URL & http://aliceyou.air-nifty.com/onesan/ \\
初版発行日 & 2019年8月12日 \\
印刷所 & Kinkos  \\ \bottomrule
\end{tabular}